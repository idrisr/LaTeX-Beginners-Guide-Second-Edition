\documentclass{article}
\begin{document}
\section*{Quadratic equations}
The quadratic equation
\begin{equation}
  \label{quad}
  ax^2 + bx + c = 0,
\end{equation}
where \( a, b \) and \( c \) are constants and \( a \neq 0 \),
has two solutions for the variable \( x \):
\begin{equation}
  \label{root}
  x_{1,2} = \frac{-b \pm \sqrt{b^2-4ac}}{2a}. 
\end{equation}
If the \emph{discriminant} \( \Delta \) with
\[
  \Delta = b^2 - 4ac
\]
is zero, then the equation (\ref{quad}) has a double solution:
(\ref{root}) becomes
\[
  x = - \frac{b}{2a}.
\]
\end{document}